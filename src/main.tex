\documentclass[11pt,a4paper]{article}

\usepackage[T1]{fontenc}
\usepackage[utf8]{inputenc}
\usepackage{lmodern}
\usepackage{geometry}
\usepackage{hyperref}
\usepackage{enumitem}
\usepackage{xcolor}
\usepackage{listings}
\usepackage{graphicx}
\usepackage{tikz}
\usepackage{svg}
\usepackage[
  backend=biber,
  style=numeric,
  sorting=ynt
]{biblatex}

\geometry{margin=2.5cm}

\definecolor{codebg}{rgb}{0.95,0.95,0.95}
\lstset{
  backgroundcolor=\color{codebg},
  basicstyle=\ttfamily\small,
  breaklines=true,
  frame=single,
  columns=fullflexible
}
\definecolor{delim}{RGB}{20,105,176}
\definecolor{numb}{RGB}{106, 109, 32}
\definecolor{string}{rgb}{0.64,0.08,0.08}
\lstdefinelanguage{json}{
    showspaces=false,
    showtabs=false,
    breaklines=true,
    postbreak=\raisebox{0ex}[0ex][0ex]{\ensuremath{\color{gray}\hookrightarrow\space}},
    breakatwhitespace=true,
    basicstyle=\ttfamily\small,
    upquote=true,
    morestring=[b]",
    stringstyle=\color{string},
    literate=
     *{0}{{{\color{numb}0}}}{1}
      {1}{{{\color{numb}1}}}{1}
      {2}{{{\color{numb}2}}}{1}
      {3}{{{\color{numb}3}}}{1}
      {4}{{{\color{numb}4}}}{1}
      {5}{{{\color{numb}5}}}{1}
      {6}{{{\color{numb}6}}}{1}
      {7}{{{\color{numb}7}}}{1}
      {8}{{{\color{numb}8}}}{1}
      {9}{{{\color{numb}9}}}{1}
      {\{}{{{\color{delim}{\{}}}}{1}
      {\}}{{{\color{delim}{\}}}}}{1}
      {[}{{{\color{delim}{[}}}}{1}
      {]}{{{\color{delim}{]}}}}{1},
}

\hypersetup{
  colorlinks=true,
  linkcolor=blue,
  urlcolor=blue
}

\bibliography{src/bibliography}

\title{Wallet Messaging Protocol}
\author{}
\date{}

\setcounter{secnumdepth}{3}
\setcounter{tocdepth}{3}

\begin{document}

  \maketitle
  \tableofcontents
  \newpage

  \section{Introduction}

  Wallet Messaging Protocol (WMP) is a websocket~\cite{websocket} based
  communication protocol which provides secure trusted channels where two
  wallets can exchange data such as OpenID4VCI~\cite{OIDC4VCI} credential offers or credential verification requests.

  It does not aim to replace these specifications, only to provide an
  additional layer through which wallets can exchange information.

  \section{Architecture}

  \subsection{Terms}

  \begin{itemize}
    \item \textbf{Entity}: Wallet communicating using WMP
    \item \textbf{Server Entity}: Entity providing WMP websocket connection
    \item \textbf{Client Entity}: Entity connecting to a WMP websocket entity
    \item \textbf{WMP Channel}: A websocket connection between two entities
    \item \textbf{Wallet Type}: \texttt{issuer}, \texttt{verifier}, \texttt{holder}
    \item \textbf{Invitation}: JWT object issued by the server
    \item \textbf{Key Identifier}: DID, X509 certificate, etc.
    \item \textbf{ID Token}: Entity description and key identifiers
  \end{itemize}

  \subsection{Invitation Flow}
  WMP connection between two previously unknown entities are established using this flow.
  The flow is described by the following diagram:

  \begin{figure}
    \centering
    \includesvg{src/figures/invitation-flow}
  \end{figure}

% ============================================================
  \subsection{Known Entity Flow}
  All messages in WMP are Json Web Tokens (JWTs)~\cite{jwt} that are signed by the corresponding entity.
  The following JWT header claims shall be present in every message:

  \begin{figure}
    \centering
    \includesvg{src/figures/known-entity-flow}
  \end{figure}

  \section{WMP Message Structure}

  All WMP messages are JSON Web Tokens (JWTs)~\cite{jwt}.
  The following JWT header claims shall be present in every message:
  \begin{enumerate}
    \item \texttt{typ}: Specifies the type of WMP message, e.g. \texttt{jwt+wmp/verification\_request}
    \item \texttt{jwk}: Public key that is used to verify the JWT signature validity.
          For now, other methods of specifying the key (such as using \texttt{kid} or \texttt{x5c} claim) are not supported
  \end{enumerate}
  The following JWT payload claims shall be present in every message:
  \begin{enumerate}
    \item \texttt{iss}: Identifier of the message issuer.
          In cases where the entity has a public URL, it shall be used.
          Otherwise, the \url{https://self-issued.me} identifier shall be used.
  \end{enumerate}

  Specific message types define additional JWT payload claims.

  \section{Invitation}

  WMP invitation is a JWT signed by a WMP server entity which is then published.
  A WMP invitation has the following JWT header claims:
  \begin{itemize}
    \item \texttt{typ}: \texttt{jwt+wmp/invitation}
  \end{itemize}

  And the following JWT Payload claims:
  \begin{itemize}
    \item \texttt{channel\_url}: A websocket url of the WMP connection
    \item \texttt{code\_challenge}: a PKCE code challenge~\cite{pkce} that will be used by the client to validate the invitation
    \item \texttt{code\_challenge\_method}: a PKCE code challenge~\cite{pkce} method.
          Only the value of \texttt{S256} is supported.
    \item \texttt{exp}: The invitation expiration.
          It is recommended to be no longer than 24 hours.
  \end{itemize}

  Here is an example of a WMP invitation:
  \begin{lstlisting}
eyJhbGciOiJFUzI1NiIsImp3ayI6eyJjcnYiOiJQLTI1NiIsImt0eSI6IkVDIiwieCI6IlpDeEF5bGR
XVDdiTHpZQjlQaXVTdTBTRGJLdFZZdnQtY3FnbmVvaDR2dnciLCJ5IjoiRmd6bkFkNzZ2NDl4OEhnU1
Q3NmdMcVBSV3JSU0tQYTA3ZDBkUUFxZG9HNCJ9LCJ0eXAiOiJqd3Qrd21wL2ludml0YXRpb24ifQ.ey
JjaGFubmVsX3VybCI6IndzOi8vbG9jYWxob3N0OjEyMzQvYXBpL3YxL3dtcC9jaGFubmVsLzRlMDU4M
2ZlLTVhZTAtNDYyMy04MzI4LTM3YTYxNDNiMzZkZC8wMTk5ZjkxNi1mNDA3LTcyNWYtOWQwMC0wM2Fi
YTA3ZTlmMDEiLCJjb2RlX2NoYWxsZW5nZSI6ImVzdTZlMm9DU1Ayd2VGRTVsUXlTdG5VV0JZOTd3MVg
yeXRSYm4yYlBIYjAiLCJjb2RlX2NoYWxsZW5nZV9tZXRob2QiOiJTMjU2IiwiZXhwIjoxNzYwOTA3MD
U1LCJpYXQiOjE3NjA4MjA2NTUsImlzcyI6IndzOi8vbG9jYWxob3N0OjEyMzQvYXBpL3YxL3dtcC80Z
TA1ODNmZS01YWUwLTQ2MjMtODMyOC0zN2E2MTQzYjM2ZGQifQ.jhesEZ4Ym5u5fVdV2f8hxECXuMI-4
RE8uIfpU--xRw2lcwirCF0ZHL9qq8JK-hSZ_YOoJEe2E7JWZsduWvWFIA
  \end{lstlisting}
  Here is the decoded token header:
  \begin{lstlisting}[language=json]
{
  "alg": "ES256",
  "jwk": {
    "crv": "P-256",
    "kty": "EC",
    "x": "ZCxAyldWT7bLzYB9PiuSu0SDbKtVYvt-cqgneoh4vvw",
    "y": "FgznAd76v49x8HgST76gLqPRWrRSKPa07d0dQAqdoG4"
  },
  "typ": "jwt+wmp/invitation"
}
  \end{lstlisting}

  Here is the decoded token payload:
  \begin{lstlisting}[language=json]
{
  "channel_url": "ws://localhost:1234/api/v1/wmp/channel/4e0583fe-5ae0-4623-8328-37a6143b36dd/0199f916-f407-725f-9d00-03aba07e9f01",
  "code_challenge": "esu6e2oCSP2weFE5lQyStnUWBY97w1X2ytRbn2bPHb0",
  "code_challenge_method": "S256",
  "exp": 1760907055,
  "iat": 1760820655,
  "iss": "ws://localhost:1234/api/v1/wmp/4e0583fe-5ae0-4623-8328-37a6143b36dd"
}
  \end{lstlisting}

  The invitation shall be published by a WMP Server Entity to an endpoint where it can be retrieved by a Client with an HTTP GET request.
  This URL can be delivered to the client by other channels such as QR code, email, processes described by RFC-1149~\cite{pigeons}, etc.

  \section{WMP Messages}
  In addition to the specified message types, more can be added to extend the functionality of WMP.

  \subsection{Verification Request}
  Verification request is sent by the WMP Server Entity to the Client Entity.
  It's header \texttt{typ} claim shall be set to \texttt{jwt+wmp/verification\_request}.
  The payload shall contain the following claims:
  \begin{itemize}
    \item \texttt{nonce}: A random string that the client has to include in a verification response.
      A source of randomness with high entropy is recommended.
    \item \texttt{code\_verifier}: Code verifier to the PKCE Code challenge~\cite{pkce} present in the invitation.
      Present only during the invitation flow.
  \end{itemize}
  Here is an example of a WMP Verification Request:
  \begin{lstlisting}
eyJhbGciOiJFUzI1NiIsImp3ayI6eyJjcnYiOiJQLTI1NiIsImt0eSI6IkVDIiwieCI6IlpDeEF5bGR
XVDdiTHpZQjlQaXVTdTBTRGJLdFZZdnQtY3FnbmVvaDR2dnciLCJ5IjoiRmd6bkFkNzZ2NDl4OEhnU1
Q3NmdMcVBSV3JSU0tQYTA3ZDBkUUFxZG9HNCJ9LCJ0eXAiOiJqd3Qrd21wL3ZlcmlmaWNhdGlvbl9yZ
XF1ZXN0In0.eyJjb2RlX3ZlcmlmaWVyIjoiTXl5SjRmTUNvT1pEMGMxNTJwdy1KSzFkN3I5aTZsTG9x
bnpsS0NwZXNsbyIsImV4cCI6MTc2MDgyMDk2OCwiaWF0IjoxNzYwODIwNjY4LCJpc3MiOiJodHRwOi8
vbG9jYWxob3N0OjEyMzQvYXBpL3YxL3dtcC80ZTA1ODNmZS01YWUwLTQ2MjMtODMyOC0zN2E2MTQzYj
M2ZGQiLCJub25jZSI6IjZHajZVLW1QSnlXU2hnbENfb3V3ZEd3SUhJQjhySk1sek5QYmVXSnp4MUEif
Q.58pZe7b5QQYRkB111UcFwxHaGR3k8pZ8TiFShfLYFfj-ZAx8cSFYEDQjwo9yrE2aFRx0x6hbo7WgL
HGyNkNvug
  \end{lstlisting}

  Here is the decoded token header:
  \begin{lstlisting}[language=json]
{
  "alg": "ES256",
  "jwk": {
    "crv": "P-256",
    "kty": "EC",
    "x": "ZCxAyldWT7bLzYB9PiuSu0SDbKtVYvt-cqgneoh4vvw",
    "y": "FgznAd76v49x8HgST76gLqPRWrRSKPa07d0dQAqdoG4"
  },
  "typ": "jwt+wmp/verification_request"
}
  \end{lstlisting}
  Here is the decoded token payload:
  \begin{lstlisting}[language=json]
{
  "code_verifier": "MyyJ4fMCoOZD0c152pw-JK1d7r9i6lLoqnzlKCpeslo",
  "exp": 1760820968,
  "iat": 1760820668,
  "iss": "http://localhost:1234/api/v1/wmp/4e0583fe-5ae0-4623-8328-37a6143b36dd",
  "nonce": "6Gj6U-mPJyWShglC_ouwdGwIHIB8rJMlzNPbeWJzx1A"
}
  \end{lstlisting}
  \subsection{Verification Response}
  Verification response is sent by the WMP Client Entity to the Server Entity.
  It's header \texttt{typ} claim shall be set to \texttt{jwt+wmp/verification\_response}.
  The payload shall contain the following claims:
  \begin{itemize}
    \item \texttt{nonce}
  \end{itemize}

  Here is an example verification response:
  \begin{lstlisting}
eyJhbGciOiJFUzI1NiIsImp3ayI6eyJjcnYiOiJQLTI1NiIsImt0eSI6IkVDIiwieCI6IlpDeEF5bGR
XVDdiTHpZQjlQaXVTdTBTRGJLdFZZdnQtY3FnbmVvaDR2dnciLCJ5IjoiRmd6bkFkNzZ2NDl4OEhnU1
Q3NmdMcVBSV3JSU0tQYTA3ZDBkUUFxZG9HNCJ9LCJ0eXAiOiJqd3Qrd21wL3ZlcmlmaWNhdGlvbl9yZ
XNwb25zZSJ9.eyJpYXQiOjE3NjA4MjA2NzksImlzcyI6Imh0dHA6Ly9sb2NhbGhvc3Q6NTY3OC9hcGk
vdjEvd21wLzRlMDU4M2ZlLTVhZTAtNDYyMy04MzI4LTM3YTYxNDNiMzZkZCIsIm5vbmNlIjoiNkdqNl
UtbVBKeVdTaGdsQ19vdXdkR3dJSElCOHJKTWx6TlBiZVdKengxQSJ9.pnT5cOEypNeC-5aewxB6lif_
jK7vWQ92tix5ASerGLwEbioUB84iVdCRB9P951HTFBKaJrTxLtfD4JUze9zSeg
  \end{lstlisting}

  Here is the decoded token header:
  \begin{lstlisting}[language=json]
{
  "alg": "ES256",
  "jwk": {
    "crv": "P-256",
    "kty": "EC",
    "x": "ZCxAyldWT7bLzYB9PiuSu0SDbKtVYvt-cqgneoh4vvw",
    "y": "FgznAd76v49x8HgST76gLqPRWrRSKPa07d0dQAqdoG4"
  },
  "typ": "jwt+wmp/verification_response"
}
  \end{lstlisting}
  Here is the decoded token payload:
  \begin{lstlisting}[language=json]
{
  "iat": 1760820679,
  "iss": "http://localhost:5678/api/v1/wmp/4e0583fe-5ae0-4623-8328-37a6143b36dd",
  "nonce": "6Gj6U-mPJyWShglC_ouwdGwIHIB8rJMlzNPbeWJzx1A"
}
  \end{lstlisting}

  \subsection{ID Token}

  ID token is sent by both client and server entities.
  It's header \texttt{typ} claim shall be set to \texttt{jwt+wmp/id\_token}.
  The payload shall contain the following claims:
  \begin{itemize}
    \item \texttt{entity\_types}: Array of wallet types
    \item \texttt{identifiers}: Array of wallet key identifiers
    \item \texttt{issuer\_url}: OIDC Issuer endpoint of the entity wallet.
          Present only if `entity\_types` contains `issuer`
    \item \texttt{authorization\_url}: OIDC Authorization endpoint of the entity wallet.
          Present only if `entity\_types` contains `issuer` or `verifier`
    \item \texttt{name}: OPTIONAL A human-readable identifier of the entity
  \end{itemize}
  \subsubsection{Wallet Key Identifier}
  A wallet key identifier object is a JSON object containing the following claims:
  \begin{itemize}
    \item \texttt{type}: Type of the Wallet Key Identifier.
          Currently supported types are \texttt{did} for DIDs~\cite{did} and \texttt{x509} for X509 certificate chains~\cite{x509}.
    \item \texttt{identifier}: String array.
          When \texttt{type} is set to \texttt{x509}, it's elements are base64 encoded X509 certificate chain elements with its leaf being the first element.
          When \texttt{type} is set to \texttt{did}, the array has only one element - the DID.
    \item \texttt{proof}: A sample JWT token signed with a private key corresponding to the public key that the key identifier points to.
          The JWT shall have its header \texttt{typ} claim set to \texttt{jwt+wmp/entity-id-proof}.
  \end{itemize}

  Here is an example ID Token:
  \begin{lstlisting}
eyJhbGciOiJFUzI1NiIsImp3ayI6eyJjcnYiOiJQLTI1NiIsImt0eSI6IkVDIiwieCI6IjBPRnRxUUw
3bmxxZl9zVHBxNTY1TlFEUm1sTVhVT1lQRnBQZ3p5MDFVaVkiLCJ5IjoiTGtmbTJ3T0I4THpweWlkb3
B3ZG96ck9yWTRUejNxZU84bkZVYUcxSm4wQSJ9LCJ0eXAiOiJqd3Qrd21wL2lkX3Rva2VuIn0.eyJhd
XRob3JpemF0aW9uX3VybCI6Imh0dHA6Ly9sb2NhbGhvc3Q6NTY3OC9hcGkvdjEvYXV0aC9lOTYxNGM2
NS0zNDY4LTQwMDUtYmVhMi00MjJiOGNjNDZhMzciLCJlbnRpdHlfdHlwZXMiOlsiaG9sZGVyIiwiaXN
zdWVyIl0sImlhdCI6MTc2MDgyMDE4MCwiaWRlbnRpZmllcnMiOlt7ImlkZW50aWZpZXIiOlsiZGlkOm
tleTp6MmRtekQ4MWNnUHg4VmtpN0pidXVNbUZZcldQZ1lveXR5a1VaM2V5cWh0MWo5S2JuUFVRRFJuQ
zFubVdxbWRIVTc0VXpEUTYxUTR6Z3lEZzlWeGE4amR4d1RVWERDQ2pZN2k3bUNGUkR3MnRGYjIyRTlu
MXlxRm5ya2RGWGVETnY1S2U4b1BGTVU5S3kySlVmdE1QR1M3S0RaZjRDWUx6M3Y1WjNrSlNtcXo5UzF
wMXY0Il0sInByb29mIjoiZXlKaGJHY2lPaUpGVXpJMU5pSXNJbXAzYXlJNmV5SmpjbllpT2lKUUxUST
FOaUlzSW10MGVTSTZJa1ZESWl3aWVDSTZJakJQUm5SeFVVdzNibXh4Wmw5elZIQnhOVFkxVGxGRVVtM
XNUVmhWVDFsUVJuQlFaM3A1TURGVmFWa2lMQ0o1SWpvaVRHdG1iVEozVDBJNFRIcHdlV2xrYjNCM1pH
OTZjazl5V1RSVWVqTnhaVTg0YmtaVllVY3hTbTR3UVNKOUxDSjBlWEFpT2lKcWQzUXJkMjF3TDJWdWR
HbDBlUzFwWkMxd2NtOXZaaUo5LmV5SnBZWFFpT2pFM05qQTRNakF4T0RBc0ltbHpjeUk2SW5kek9pOH
ZhSFIwY0RvdkwyeHZZMkZzYUc5emREbzFOamM0TDJGd2FTOTJNUzkzYlhBdlpUazJNVFJqTmpVdE16U
TJPQzAwTURBMUxXSmxZVEl0TkRJeVlqaGpZelEyWVRNM0luMC5NVmRmV0xQN0ZpUmNmdjhMSkJCdXJj
QS1aUFdJam1lMnd0Y182Vl8yVjU5Q250Z0d0UnFoMHhDaVJoeU9sYjFIcFkzRjZ2MHZvTE84YkVwUU9
KT01GUSIsInR5cGUiOiJkaWQifV0sImlzcyI6IndzOi8vaHR0cDovL2xvY2FsaG9zdDo1Njc4L2FwaS
92MS93bXAvZTk2MTRjNjUtMzQ2OC00MDA1LWJlYTItNDIyYjhjYzQ2YTM3IiwiaXNzdWVyX3VybCI6I
mh0dHA6Ly9sb2NhbGhvc3Q6NTY3OC9hcGkvdjEvaXNzdWVyL2U5NjE0YzY1LTM0NjgtNDAwNS1iZWEy
LTQyMmI4Y2M0NmEzNyIsIm5hbWUiOiJ0ZXN0LXdhbGxldCJ9.VGsqBqARbz1d45SpL2Qc2ZLzaZZ3AC
WKp63MfLbtkbQ-vUYDo6I4e2d1FaOMTvAS7pG5TaKiFRiiNS13sNJh6A
  \end{lstlisting}
  Here is the decoded token header:
  \begin{lstlisting}[language=json]
{
  "alg": "ES256",
  "jwk": {
    "crv": "P-256",
    "kty": "EC",
    "x": "ZCxAyldWT7bLzYB9PiuSu0SDbKtVYvt-cqgneoh4vvw",
    "y": "FgznAd76v49x8HgST76gLqPRWrRSKPa07d0dQAqdoG4"
  },
  "typ": "jwt+wmp/id_token"
}
  \end{lstlisting}
  Here is the decoded token payload:
  \begin{lstlisting}[language=json]
{
  "authorization_url": "http://localhost:5678/api/v1/auth/4e0583fe-5ae0-4623-8328-37a6143b36dd",
  "entity_types": [
    "holder",
    "issuer"
  ],
  "iat": 1760820693,
  "identifiers": [
    {
      "identifier": [
        "did:key:z2dmzD81cgPx8Vki7JbuuMmFYrWPgYoytykUZ3eyqht1j9KbqfP8PC5rxKQNVs
         WdUfonBBcEisrMdFZ8iAMi9FLEFnWmKZamRQioiqxc7KRwYDbTMgvfJN6gcoB4XM6R83wh
         vnb7kehnCEDCr5ce6FSdehHd1uBYSZTrGsN5tmDuDciBzL"
      ],
      "proof":
          "eyJhbGciOiJFUzI1NiIsImp3ayI6eyJjcnYiOiJQLTI1NiIsImt0eSI6IkVDIiwieCI6I
          lpDeEF5bGRXVDdiTHpZQjlQaXVTdTBTRGJLdFZZdnQtY3FnbmVvaDR2dnciLCJ5IjoiRm
          d6bkFkNzZ2NDl4OEhnU1Q3NmdMcVBSV3JSU0tQYTA3ZDBkUUFxZG9HNCJ9LCJ0eXAiOiJ
          qd3Qrd21wL2VudGl0eS1pZC1wcm9vZiJ9.eyJpYXQiOjE3NjA4MjA2OTMsImlzcyI6Imh
          0dHA6Ly9sb2NhbGhvc3Q6NTY3OC9hcGkvdjEvd21wLzRlMDU4M2ZlLTVhZTAtNDYyMy04
          MzI4LTM3YTYxNDNiMzZkZCJ9.QJ-QItKWSBbF0j9kOZY130AojuDgTtVr8lB8USUr4aQP
          z8LXZXUl94yS9mLNlRSBGpfYb1xUS3GDsqc49WRj2A",
      "type": "did"
    }
  ],
  "iss": "http://localhost:5678/api/v1/wmp/4e0583fe-5ae0-4623-8328-37a6143b36dd",
  "issuer_url": "http://localhost:5678/api/v1/issuer/4e0583fe-5ae0-4623-8328-37a6143b36dd",
  "name": "test-wallet"
}
  \end{lstlisting}

  \section{Credential Offer}
  Credential offer message is sent from an entity with \texttt{issuer} wallet type to an entity with \texttt{holder} wallet type.
  It's header \texttt{typ} claim shall be set to \texttt{jwt+wmp/credential\_offer}.
  The payload shall contain the following claims:
  \begin{itemize}
    \item \texttt{offer\_url}: An OIDC4VCI~\cite{OIDC4VCI} Credential Offer URL
  \end{itemize}

  Upon receiving a credential offer message, the recipient shall process it using flows specified in OIDC4VCI.

  \section{Credential Verification Request}
  Credential verification request is sent from an entity with \texttt{verifier} wallet type to an entity with \texttt{holder} wallet type.
  It's header \texttt{typ} claim shall be set to \texttt{jwt+wmp/credential\_verification\_request}.
  The payload shall contain the following claims:
  \begin{itemize}
    \item \texttt{verification\_request\_url}: An OIDC4VP~\cite{OIDC4VP} Credential Verification Request URL
  \end{itemize}

  \section{Verifying WMP Messages}
  All WMP messages are JWT tokens and before any other verification, their signature must be verified.
  Both \texttt{client} and \texttt{server} entities must also verify if the key used to sign the message is same as the key the entity used during the invitation process.

  \subsection{Verifying Invitations}
  Before establishing any websocket connection with the server, the client must verify the validity of the JWT signature.
  If the invitation endpoint is protected by a hashlink~\cite{hl}, it must also verify whether the hashlink matches the HTTP response body.
  After establishing a websocket connection, the client must validate the \texttt{code\_challenge} with the \texttt{code\_verifier}
  provided in the client Verification Request sent by the server.

  \subsection{Verifying Verification Response}
  After validating the JWT signature, the server must verify the values of the sent \texttt{nonce} claim.
  If the claim is not present, or it's value does not match the one sent by the server,
  the server must not continue communicating through this channel.

  \subsection{Verifying ID Token}
  After validating the JWT signature, the entity must verify the \texttt{proof} claim in every element of the `identifiers` claims.

  \section{Key Rotation}
  TBD

  \section{Websocket Ping \& Pong}
  TBD

  \printbibliography

\end{document}
